\documentclass[12pt]{amsart}
\usepackage{amsmath}
\usepackage{amsthm}
\usepackage{amsfonts}
\usepackage{amssymb}
\usepackage[margin=1in]{geometry}
\usepackage{hyperref}
\hypersetup{
    colorlinks=true,
    linkcolor=blue
}

\theoremstyle{definition}
\newtheorem{theorem}{Theorem}[section]
\newtheorem{lemma}[theorem]{Lemma}
\newtheorem{definition}[theorem]{Definition}
\newtheorem{corollary}[theorem]{Corollary}
\newtheorem{proposition}[theorem]{Proposition}
\newtheorem{conjecture}[theorem]{Conjecture}
\newtheorem{remark}[theorem]{Remark}
\newtheorem{example}[theorem]{Example}
\newtheorem{problem}[theorem]{Problem}
\newtheorem{notation}[theorem]{Notation}
\newtheorem{question}[theorem]{Question}
\newtheorem{caution}[theorem]{Caution}

\begin{document}

\title{Homework 5}

\maketitle

For this week, please answer the following questions from the text. 
I've copied the problem itself below and the question numbers for 
your convenience. 

\begin{enumerate}
	\item (2.6) Alice and Bob agree to use the prime $p=1373$ and 
		the base $g=2$ for the Diffie-Hellman key exchange. Alice 
		sends Bob the value $A = 974$. Bob asks your assistance, 
		so you tell him to use the secret exponent $b=871$. What 
		value $B$ should Bob send to Alice, and what is their 
		secret shared value? Can you figure out Alice's secret 
		exponent?
	\item (2.7) Let $p$ be a prime and let $g$ be an integer. The 
		\textit{Decision Diffie-Hellman Problem} is as follows. 
		Suppose that you are given three numbers $A$, $B$, and $C$, 
		and suppose that $A$ and $B$ are equalt to 
		\begin{displaymath}
			A = g^a \mod p, \ B = g^b \mod p
		\end{displaymath}
		but that you do not necessarily know the exponents $a$ and 
		$b$. Determine whether $C$ is equal to $g^{ab} \mod p$. Notice 
		that this is different from the Diffie-Hellman problem itself. 
		\begin{enumerate}
			\item Prove that an algorithm that solves the Diffie-
				Hellman problem can be used to solve the decision 
				Diffie-Hellman problem.
			\item Do you think that the decision Diffie-Hellman problem 
				is hard or easy? Why? 
		\end{enumerate}
	\item (2.8) Alice and Bob agree to use the prime $p=1373$ and the base 
		$g=2$ for communications using the Elgamal public key cryptosystem. 
		\begin{enumerate}
			\item Alice chooses $a=947$ as her private key. What is the 
				value of her public key $A$?
			\item Bob choose $b=716$ as his private key, so his public 
				key is 
				\begin{displaymath}
					B = 2^{716} = 469 \mod 1373
				\end{displaymath}
				Alice encrypts the message $m=583$ using the random 
				element $k=877$. What is the ciphertext $(c_1,c_2)$ 
				that Alice sends to Bob?
			\item Alice decides to choose a new private key $a=299$ with 
				associated public key $A = 2^{299} = 34 \mod 1373$. 
				Bob encrypts a message using Alice's public key and 
				sends her the ciphertext $(c_1,c_2) = (661,1325)$. 
				Decrypt this message. 
			\item Now Bob choose a new private key and publishes the 
				associated public key $B = 893$. Alice encrypts a 
				message using the this public key and sends the ciphertext 
				$(c_1,c_2) = (693,793)$ to Bob. Eve intercepts the 
				transmission. Help Eve by the solving the discrete 
				logarithm problem $2^b = 893 \mod 1373$ and using the 
				value of $b$ to decrypt the message. 
		\end{enumerate}
	\item (2.9) Suppose Eve is able to solve the Diffie-Hellman problem. More precisely, 
		assume that Eve is given two powers $g^u$ and $g^v \mod p$, then she is 
		able to compute $g^{uv} \mod p$. Show that Eve can break the Elgamal PKC.
	\item (2.10) This exercise describes a public key cryptosystem that requires Bob and 
		Alice to exchange several messages. We illustrate the system with an example. 

		Bob and Alice fix a publicly known prime $p=32611$, and all of the other 
		numbers are private. Alice takes her message $m=11111$, choose a random 
		exponent $a=3589$, and sends the number $u = m^a \mod p = 15950$ to Bob. Bob 
		chooses a random exponent $b=4037$ and sends $v = u^b \mod p = 15422$ back 
		to Alice. Alice then computes $w = v^{15619} = 27257 \mod 32611$ and sends 
		$w = 27257$ to Bob. Finally, Bob computes $w^{31883} mod 32611$ and recovers 
		the value $11111$ of Alice's message. 
		\begin{enumerate}
			\item Explain why this algorithm works. In particular, Alice uses the 
				numbers $a=3589$ and $15619$ as exponents. How are they related? 
				Similarly, how are Bob's exponents $b=4037$ and $31883$ relates? 
			\item Formulate a general version of this cryptosystem, i.e. using variables, 
				and show that it works in general. 
			\item What is the disadvantage of this cryptosystem over Elgamal? (Hint: 
				How many times must Alice and Bob exchange data?)
			\item Are there any advantages of this cryptosystem over Elgamal? In particular, 
				can Eve break it if she can solve the discrete logarithm problem? 
				Can Eve break it if she can solve the Diffie-Hellman problem?
		\end{enumerate}
\end{enumerate}

\end{document}
